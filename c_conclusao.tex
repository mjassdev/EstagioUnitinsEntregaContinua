\chapter{Conclusão}\label{cap:conclusao}

Este trabalho concentrou-se na apresentação dos conceitos da visão DevOps e analogia com o modelo tradicional de entrega de um software, onde ao decorrer do estudo analisou-se diversas ferramentas disponíveis no mercado e elegeram-se as que mais se adequavam a necessidade da implementação de um modelo de entrega para aplicação externa com o conceito de deploy automatizado.

O deploy automatizado fornece ao desenvolvedor e profissional de TI um ambiente simplificado para entrega de uma aplicação externa, tornando o gestor de tecnologia um provedor de serviço.

O presente modelo tem como foco contribuir para o aprimoramento no processo de entrega de software. Nesse sentido o cenário atual ainda lida com o processo de deploy baseado no modelo tradicional "cascata", com o estudo realizado nesse projeto é possível otimizar o planejamento do processo da entrega da aplicação para produção utilizando recursos automatizados e gratuitos (open source) e uma visão DevOps. É possível economizar não só tempo como recursos humanos e financeiros que são fundamentais para a conclusão do processo.

Para continuidade deste projeto, objetiva-se a implementação de um cenário funcional utilizando máquinas virtuais, como expõe-se a seguir:

\begin{enumerate}
	\item Implementar o cenário apresentado neste projeto;
	\item Validar o modelo considerando diversas situações;
	\item Implementar o modelo visando ambientes críticos e cenários de risco.
	\item Implantar a solução afim de gerar um estudo de caso, visando verificar a aceitação da mesma.
\end{enumerate}