
% ------------------------------------------------------------------------
% ------------------------------------------------------------------------
% abnTeX2: Modelo de Trabalho Academico (tese de doutorado, dissertacao de
% mestrado e trabalhos monograficos em geral) em conformidade com 
% ABNT NBR 14724:2011: Informacao e documentacao - Trabalhos academicos -
% Apresentacao
% ------------------------------------------------------------------------
% ------------------------------------------------------------------------


%Testando o git

\documentclass[
	% -- opções da classe memoir --
	12pt,				% tamanho da fonte
	openright,			% capítulos começam em pág ímpar (insere página vazia caso preciso)
	oneside,			% para impressão em verso e anverso. Oposto a oneside
	a4paper,			% tamanho do papel. 
	% -- opções da classe abntex2 --
	%chapter=TITLE,		% títulos de capítulos convertidos em letras maiúsculas
	%section=TITLE,		% títulos de seções convertidos em letras maiúsculas
	%subsection=TITLE,	% títulos de subseções convertidos em letras maiúsculas
	%subsubsection=TITLE,% títulos de subsubseções convertidos em letras maiúsculas
	% -- opções do pacote babel --
	english,			% idioma adicional para hifenização
	brazil				% o último idioma é o principal do documento
	]{abntex2}

% ---
% Pacotes básicos 
% ---
\usepackage{unitins}
% \usepackage{lmodern}			% Usa a fonte Latin Modern			
% \usepackage[T1]{fontenc}		% Selecao de codigos de fonte.
\usepackage[utf8]{inputenc}		% Codificacao do documento (conversão automática dos acentos)
\usepackage{lastpage}			% Usado pela Ficha catalográfica
\usepackage{indentfirst}		% Indenta o primeiro parágrafo de cada seção.
\usepackage{color}				% Controle das cores

\usepackage{graphicx}			% Inclusão de gráficos
\usepackage{subfig}
\usepackage{microtype} 			% para melhorias de justificação
\usepackage{soul}
\usepackage{amssymb}
\usepackage{amsmath}
\usepackage{spreadtab}
\usepackage{multirow}
\usepackage{amsthm}
\usepackage{url}
% \usepackage[portuguese, ruled, linesnumbered]{algorithm2e}


% ---
		
% ---
% Pacotes adicionais, usados apenas no âmbito do Modelo Canônico do abnteX2
% ---
		% para geração de dummy text
% ---

% ---
% Pacotes de citações
% ---

%\usepackage[brazilian,hyperpageref]{backref}	 % Paginas com as citações na bibl
\usepackage[alf,abnt-etal-list=2 ]{abntex2cite}	% Citações padrão ABNT


\usepackage[table]{xcolor}
\definecolor{lightgray}{gray}{0.9}
\graphicspath{{imagens/}}

\theoremstyle{theorem}
\newtheorem{teo}{Teorema}[chapter]
\newtheorem{lema}[teo]{Lema}
\theoremstyle{definition}
\newtheorem{defi}[teo]{Definição}

% --- 
% CONFIGURAÇÕES DE PACOTES
% --- 

% ---
% Configurações do pacote backref
% Usado sem a opção hyperpageref de backref
%\renewcommand{\backrefpagesname}{Citado na(s) página(s):~}
% Texto padrão antes do número das páginas
%\renewcommand{\backref}{}
% Define os textos da citação
%\renewcommand*{\backrefalt}[4]{
%	\ifcase #1 %
%		Nenhuma citação no texto.%
%	\or
%		Citado na página #2.%
%	\else
%		Citado #1 vezes nas páginas #2.%
%	\fi}%
% ---

% ---
% Informações de dados para CAPA e FOLHA DE ROSTO
% ---
\titulo{PROPOSTA DE UM MODELO DE IMPLANTAÇÃO PARA APLICAÇÃO EXTERNA UTILIZANDO CONCEITOS DEVOPS}
\autor{MATHEUS JOSÉ ALVES SILVA SANTOS}
\local{Palmas}
\data{2018}
\orientador{Prof. Me. Douglas Chagas da Silva}
\instituicao{%
  Universidade Estadual do Tocantins
  \par
  Curso de Sistemas de Informação
  }
\tipotrabalho{Trabalho de Conclusão de Curso (Graduação)}
% O preambulo deve conter o tipo do trabalho, o objetivo, 
% o nome da instituição e a área de concentração 
\preambulo{Projeto apresentado como requisito para aprovação na disciplina de Estágio Supervisionado do Curso de Sistemas de Informações da Universidade Estadual do Tocantins- UNITINS, sob a orientação do professor Me. Douglas Chagas da Silva.}

% ---


% ---
% Configurações de aparência do PDF final

% alterando o aspecto da cor azul
\definecolor{blue}{RGB}{41,5,195}

% informações do PDF
\makeatletter
\hypersetup{
     	%pagebackref=true,
		pdftitle={\@title}, 
		pdfauthor={\@author},
    	pdfsubject={\imprimirpreambulo},
	    pdfcreator={LaTeX with abnTeX2},
		pdfkeywords={Algoritmo}{trabalho acadêmico}, 
		colorlinks=true,       		% false: boxed links; true: colored links
    	linkcolor=blue,          	% color of internal links
    	citecolor=blue,        		% color of links to bibliography
    	filecolor=magenta,      		% color of file links
		urlcolor=blue,
		bookmarksdepth=4
}
\makeatother
% --- 

% --- 
% Espaçamentos entre linhas e parágrafos 
% --- 

% O tamanho do parágrafo é dado por:
\setlength{\parindent}{1.3cm}

% Controle do espaçamento entre um parágrafo e outro:
\setlength{\parskip}{0.2cm}  % tente também \onelineskip

% ---
% compila o indice
% ---
\makeindex
% ---

% ----
% Início do documento
% ----
\begin{document}

% Retira espaço extra obsoleto entre as frases.
\frenchspacing 

% ----------------------------------------------------------
% ELEMENTOS PRÉ-TEXTUAIS
% ----------------------------------------------------------
% \pretextual
% ----------------------------------------------------------

% ----------------------------------------------------------
% ELEMENTOS PRÉ-TEXTUAIS
% ----------------------------------------------------------
% \pretextual

% ---
% Capa
% ---
\imprimircapa
% ---

% ---
% Folha de rosto
% (o * indica que haverá a ficha bibliográfica)
% ---
\imprimirfolhaderosto
% ---

% ---
% Inserir folha de aprovação
% ---

% Isto é um exemplo de Folha de aprovação, elemento obrigatório da NBR
% 14724/2011 (seção 4.2.1.3). Você pode utilizar este modelo até a aprovação
% do trabalho. Após isso, substitua todo o conteúdo deste arquivo por uma
% imagem da página assinada pela banca com o comando abaixo:
%
% \includepdf{folhadeaprovacao_final.pdf}
%
\begin{folhadeaprovacao}

  		\includegraphics[width=1\textwidth]{imagens/unitins.png}
  		
  		\ABNTEXchapterfont\large   1. INFORMAÇÕES DO ACADÊMICO:
				
  		\normalsize Nome: Matheus José Alves Silva Santos
  		\tab Matrícula:
  
  		Período 7	
  		
  		E-mail: matheusetf@gmail.com
  		\tab Telefone: (63) 99285-7729
  		
  		\par
  		\vspace*{0.5cm}     
  		  
  		\ABNTEXchapterfont\large   2. INFORMAÇÕES DO ESTÁGIO:
		  		\normalsize 
		
		Professor Orientador: Douglas Chagas da Silva
		
		Início do Estágio: \tab \tab Término do Estágio: 16/06/2018
		
		Total de horas semanais dedicadas ao estágio supervisionado: 20 horas
		
		Área de realização do estágio: Infraestrutura Ágil
		
		
		Data: \_\_\_\_$/$\_\_\_\_$/$\_\_\_\_
		\par
		\vspace*{0.5cm}

  		\ABNTEXchapterfont\large   3. PARECER DO PROFESSOR ORIENTADOR:
  		\normalsize 
  		
  		Aprovado $($ \tab $)$ \tab \tab Reprovado $($ \tab $)$  \tab \tab Nota: \_\_\_\_\_
  		
  	 OBSERVAÇÕES:
  		  		
  		  		 DATA: \_\_\_\_$/$\_\_\_\_$/$\_\_\_\_ \tab \assinatura{Professor Orientador}
  		\par
  		\vspace*{0.5cm}
  		 \ABNTEXchapterfont\large   4. PARECER DO COORDENADOR DE ESTÁGIO:
  		   		\normalsize 
  		
  		 OBSERVAÇÕES: 

  		 DATA: \_\_\_\_$/$\_\_\_\_$/$\_\_\_\_ \tab \assinatura{Coord. de Estágio Supervisionado}
  		 
   \begin{center}
    \vspace*{0.1cm}
    {\large\imprimirlocal}
    \par
    {\large\imprimirdata}
    \vspace*{0.5cm}
  \end{center}
  
\end{folhadeaprovacao}
% ---

% ---
% Dedicatória
% ---
\begin{dedicatoria}
   \vspace*{\fill}
   \centering
   \noindent
   \textit{ Este trabalho é dedicado à minha família, pelo apoio incondicional.} \vspace*{\fill}
\end{dedicatoria}
% ---

% ---
% Agradecimentos
% --- Es
\begin{agradecimentos}
À Deus, pela graça de permitir o meu crescimento no campo de estudo em que me sinto feliz e realizado. \\

Aos meus pais e meus irmãos que sempre apostaram em minha capacidade de crescimento e superar as barreiras que a vida impõe.\\

À minha esposa que tem sido compreensiva nos momentos de dificuldades, e companheira nos desafios que surgem.\\

Aos meus amigos e colegas de universidade, que sempre contribuíram para o desenvolvimento do meu conhecimento e pelas parcerias nos projetos e estudos.


\end{agradecimentos}
% ---

% ---
% Epígrafe
% ---
\begin{epigrafe}
    \vspace*{\fill}
	\begin{flushright}
		\textit{``Não só isso, mas também nos gloriamos \\
			nas tribulações, porque sabemos que a tribulação\\
			produz perseverança; a perseverança, um caráter \\
			aprovado; e o caráter aprovado, esperança. \\
			E a esperança não nos decepciona, porque Deus \\
			derramou seu amor em nossos corações, por meio \\
			do Espírito Santo que Ele nos concedeu.``\\
			(Bíblia Sagrada, Romanos 5:3-5)}
			
	\end{flushright}
\end{epigrafe}
% ---

% ---
% RESUMOS
% ---

% resumo em português
\setlength{\absparsep}{18pt} % ajusta o espaçamento dos parágrafos do resumo
\begin{resumo}
Este projeto visa apresentar um estudo sobre as diversas ferramentas de automação da entrega de software, visando estabelecer uma proposta de cenário que viabilize esse processo dentro de uma cultura DevOps. O cenário deve funcionar como um provedor de serviços sobre plataforma ágil.
 

 \textbf{Palavras-chaves}: Automação, DevOps, Modelo Ágil.
 
\end{resumo}

% resumo em inglês



% ---
% inserir lista de ilustrações
% ---
\pdfbookmark[0]{\listfigurename}{lof}
\listoffigures*
% ---

% ---
% inserir lista de tabelas
% ---
% \newpage
% \pdfbookmark[0]{\listtablename}{lot}
% \newpage
% \listoftables*
% \cleardoublepage
% ---

% ---
% inserir lista de abreviaturas e siglas
% ---
\begin{siglas}
  \item DevOps - Development and Operational (Desenvolvimento e Operacional).
  \item AWS - Amazon Web Services
  \item SSH - Secure Shell
  \item NETCONF - Network Configuration
\end{siglas}
% ---


% ---
% inserir o sumario
% ---
\pdfbookmark[0]{\contentsname}{toc}
\tableofcontents*
\cleardoublepage
% ---



% ELEMENTOS TEXTUAIS
% ---------------------------------[!htb]-------------------------
%\textual

%Capitulos
% ----------------------------------------------------------
% Introdução (exemplo de capítulo sem numeração, mas presente no Sumário)
% ----------------------------------------------------------


\chapter{Introdução}\label{intro}

As mudanças econômicas refletem demandas de mercado. Isso têm impacto fundamental na forma como os recursos são disponibilizados e os serviços são exigidos. Nesse sentido, a tecnologia da informação, e mais especificamente nesse estudo, a entrega de um produto ou serviço deve dar um suporte a altura das cobranças de mercado.

O cenário atual que envolve os processos de desenvolvimento e infraestrutura estão ficando cada vez mais defasados. Tradicionalmente o processo de \textit{deploy}\footnote{Publicação de um determinado software ou serviço para uso.} e gestão de entrega de soluções, tende a tornar a rotina de produção e disponibilização de serviços demasiadamente morosa. Em termos atuais o usuário final e o próprio mercado demanda agilidade e rápida resposta.

A concepção de um modelo de entrega contínua que permita ganho na produtividade e fácil detecção de erros no processo de \textit{deploy}, tem sido cada vez mais requisitado no meio da gestão da tecnologia de informação aliado a uma cultura \textit{DevOps}\footnote{Development and Operational (Desenvolvimento e Operacional).}. Tomando isso como base, é importante entendermos que os prazos de entrega estão cada vez mais apertados, o que influi no aumento da carga de trabalho, refletindo diretamente no desempenho do profissional, tanto de desenvolvimento quanto de infraestrutura.

O presente projeto tem como foco a formulação de um modelo que permita essa entrega contínua e automatizada no que diz respeito o \textit{deploy} de aplicações em tempo minimamente reduzido. Passa-se antes pelo estudo das diversas ferramentas disponíveis no mercado de software, a fim de levantar-se as possibilidades de aplicação nesse modelo. Sendo assim serão apresentadas as ferramentas eleitas como mais viáveis para o projeto.

É vital então percebermos a real vantagem no uso do modelo baseado na visão DevOps, a redução na carga de trabalho e a comunicação massiva entre os times de infraestrutura e desenvolvimento dentro de um mesmo projeto, e a entrega ágil do produto trabalhado.


A estrutura desse trabalho está organizado da seguinte forma: primeiramente são traçados os objetivos. Na seção 2 será apresentado o referencial teórico utilizados para embasar o tema e a ideia proposta de acordo com todas as ferramentas estudadas a fim de utilização posterior em um cenário fictício; na seção 3 será apresentado os métodos utilizados para compor o estudo, assim como as ferramentas utilizadas durante todo o processo. Na seção 4, são apresentados os resultados referente ao projeto. Por fim, menciona-se as referências utilizados no referido estudo.

\section{Objetivos}
\subsection{Objetivo Geral }
Apresentar um modelo de arquitetura de aplicação externa para entrega contínua baseada na visão DevOps.

\begin{comment}
Este é um comentário
\end{comment}

\subsection{Objetivos Específicos}
\begin{itemize}
	
\item Diferenciar o modelo tradicional de deploys e a visão DevOps;

\item Pesquisar as ferramentas disponíveis mais relevantes que viabilizam a entrega contínua para aplicação de terceiros;

\item Definir os benefícios do uso da visão DevOps numa aplicação;

\item Apresentar o ganho real na produtividade e tempo, no uso da entrega contínua.

\end{itemize}

% ---
% Capitulo de revisão de literatura
% ---

\chapter{Referencial Teórico}\label{referencial_teorico}


O conceito de DevOps pode ser entendido como o nivelamento entre as equipes de desenvolvimento e operações no que tange suas interações, mantendo suas funções específicas, porém alinhando suas demandas referentes às responsabilidades e processos, visando a disponibilização de um produto ou funcionalidade de forma rápida e confiável.\cite{gartnerglossario}

\begin{figure}[htb] %Figura: Ciclo de vida DevOps
	\centering
	\includegraphics[width=1\linewidth]{figura1}
	\caption{Ciclo DevOps}
	Fonte: Amazon Web Services, Inc., 2018
	\label{fig:figura1}
\end{figure}

As implementações nesse tipo de ambiente fazem uso de ferramentas de automação com o intuito de dinamizar cada vez mais a infraestrutura e torná-la mais programável, de forma que reflita na melhoria contínua da comunicação e integração entre desenvolvedores e administradores de infra, transformando o cenário tradicional de isolamento entre essas duas equipes em um ambiente participativo e colaborativo.\cite{costa}

O objetivo do DevOps é gerar em toda a equipe envolvida na produção de um software, uma cultura que vise o aumento do fluxo de trabalho (maior frequência de deploys) e em paralelo a isso dar mais robustez no desenvolvimento da aplicação. Esse conceito representa muito mais do que simplesmente o uso de ferramentas de automação, é importante observarmos que trata-se de uma quebra de paradigmas e uma mudança na cultura no negócio com uma nova forma de produção.\cite{sato2014devops}

A agilidade no processo de deploys citado anteriormente aluz à uma necessidade de amparo para que essa entrega rápida de fato aconteça, e mais do que isso, a demanda por parte dos clientes ou usuários de determinada aplicação, requer cada vez mais velocidade no uso de determinada funcionalidade. Analogamente à revolução industrial do século XX, a mudança na forma de produção de produtos naturalmente está sendo absorvida pela tecnologia, assim, a concepção desse produto deve acompanhar a demanda externa. Nesse sentido, a adaptabilidade e mudança da arquitetura funcional no desenvolvimento de serviços tecnológicos são necessidades relevantes, a cultura DevOps é uma mudança importante nesse ponto.\cite{ibmdevops}

Segundo uma pesquisa realizada pelo Gartner Group sobre DevOps, em 2015, somente 29\% das organizações pesquisadas tem o modelo atuante em produção. É evidenciado ainda que apenas 42\% desses, tem a atuação do DevOps em aplicações móveis. De acordo com o mesmo grupo o DevOps evoluiria de uma estratégia de nicho para uma estratégia comum sendo empregada por 25\% das organizações do Global 2000\footnote{Forbes Global 2000 é uma classificação anual das 2.000 empresas públicas do mundo pela revista Forbes. O ranking é baseado em quatro critérios: vendas, lucro, ativos e valor de mercado. A lista é publicada desde 2003. Fonte: Forbes}.\cite{gartnerglobal}

O uso da cultura DevOps deve ser absorvida na necessidade de versionamento contínuo de determinada aplicação, ou seja, a frequente execução de deploys. Nisso, é importante observarmos que a rápida proliferação de software requer atualizações operando na mesma medida ágil e demanda pela competitividade de mercado, visto que a velocidade no atendimento da expectativa dos clientes diferencia a empresa em relação às demais atuantes no mercado.

A principal vantagem no uso de DevOps é a melhora evidente nos processos e automatização das tarefas, otimizando o tempo e reduzindo os ciclos de desenvolvimento. Por se tratar de uma interação entre as equipes de desenvolvedores e operacionais, dizemos que é um sistema bimodal de trabalho.\cite{sato2014devops}

O monitoramento de métricas e registro de logs é um aspecto relevante. Leva-se em conta que os serviços devem estar disponíveis 24 horas por dias e durante os 7 dias da semana, acarretando em uma massiva análise de dados e logs gerados pelo sistema, portanto, a rotina na observância desses elementos deve ser constante. É possível, inclusive, a criação de alertas que apontem situações e permitam a gerência proativa dos serviços.

Outro benefício fundamental do DevOps é o aumento na comunicação e colaboração que envolve todos os personagens da empresa. É importantíssimo a definição de normas que permitam um maior compartilhamento de informações, ou que permita a proliferação da comunicação, seja qual for o método ou tecnologia, desde que agregue valor. Além disso, a diminuição de ruídos na comunicação e conflitos entre as equipes melhora o ambiente e tende a produzir efeitos positivos ao fim do processo.\cite{gaea}

Podemos ainda elencar ganhos em maior estabilidade e melhor desempenho, e tão importante quanto, a redução considerável de custos de trabalho, visto que a diminuição de tempo de produção e menor esforço afeta diretamente o custo estimado em um projeto.


\section{Infraestrutura como Código}
Falar em infraestrutura com código é exatamente absorver o entendimento do tratar a estrutura de TI como um software, programável. Como isso é possível o uso de práticas que envolvam o controle por versionamento, testes automatizados, entrega contínua, entre diversos outros recursos por meio de scripts específicos.

O uso de práticas anteriores com métodos de gerenciamento de infraestutura foram válidos e deram uma base poderosa para a concepção de novas tecnologias. Atualmente, há  uma constante demanda de software cada vez mais dinâmicos e um alto índice de deploys, inclusive simultâneos, levando à necessidade de concepção de uma infraestrutura que acompanhe o ritmo de complexidade desses novos sistemas.

Segundo a Hewlett Packard\footnote{Disponível em: https://www.hpe.com/br/pt}, em seu site oficial, cita que a infraestrutura como código elimina a necessidade de criar diversos ambientes de produção de forma manual, isolada e separadamente, e/ou atuailzações de hardware e sofware. Toda essa dinâmica pode ser feita através de scripts contidos no mesmo conjunto de códigos, trazendo velocidade, economia e otimização de tempo. Nesse contexto, a infraestrutura como código traça uma linha tênue entre o código que executa a aplicação e o código que configura um ambiente, tornando um ambiente característico do DevOps.


\section{Principais Ferramentas e Recursos Tecnológicos Disponíveis}
Apesar do conceito DevOps ser recente, a gama de ferramentas que contribuem para implantação dessa cultura já se mostram bastante diversificadas. Dentre as mais comuns podemos apresentar:

\subsection{Atlas}
É uma ferramenta disponibilizada pela Hashicorp, que tem a função de unificar projetos open source para o manejo de aplicações finalizadas no desenvolvimento para a produção em qualquer que seja a infraestrutura.
As etapas do Atlas seguem cinco passos (vide figura 2), isso independe da tecnologia utilizada, sejam máquinas virtuais ou contêineres, as etapas se mantêm as mesmas. \cite{hashimoto}

\begin{figure}[htb]

	\centering
	\includegraphics[width=0.8\linewidth]{etatasAtlas}
	\caption{Etapas do Atlas}
	Fonte: Hashicorp\footnotemark
	\label{fig:etatasAtlas}


\end{figure}
	\footnotetext{Disponível em: https://www.hashicorp.com/blog/atlas-announcement}




O Atlas não é um software de caixa preta, ou seja, é possível acesso a serviços que contribuem para tanto, como Vagrant (gerencia ambientes de desenvolvimento), Packer (construção de artefatos), Terraform (implantação de Infraestrutura) e Consul (monitora os serviços em tempo real).

\subsection{Chef}

É um framework destinado à automatização para sistemas e infraestrutura em nuvem. O Chef\footnote{Disponível em https://www.chef.io} constrói, entrega e administra fazendo uso de scripts replicáveis.

O Chef tira a carga dos administradores de sistemas que focam no gerenciamento projetado para servidores autônomos, ele permite executar cetenas de instâncias de servidor em um tempo imensamente maior se comparado ao uso comum em deploys.

Para gerenciar esse tipo de configuração, o Chef, transforma a infraestrutura em código, deixando o processo mais flexível, legível pelos analistas e testável, possibilitando assim a gerência de recursos tanto localmente quanto na nuvem

Em sua topologia o Chef tem três principais componentes: o Servidor Chef, as Estações e o Nós.

O maior atrativo dessa ferramenta é o uso de "cookbooks" ou receitas, são ditas configurações plugáveis, que envolvem todas as instalações e parâmetros necessários para atender determinada aplicação no servidor ou máquina. Assim como as receitas convencionais definem uma estrutura sequencial que deve ser seguida a fim de tornar o produto final reflexo de uma ideia original, o Chef mantém um conceito semelhante a isso, onde define-se um estado desejado do sistema, desenvolvendo um código de configuração, então o ele processa esse código, une ao processo os dados do nó em questão e garante que o estado concebido seja correspondente ao estado do sistema. 

O Chef pode ser executado em várias plataformas, como Windows, distribuições Linux, FreeBSD, Solaris, AIX, Cisco IO e Nexus. E ainda suporta plataformas em nuvem,como Amazon Web Services (AWS), Google Cloud Platform, OpenStack, IBM Bluemix, HPE Cloud, Microsoft Azure e VMware vRealize.

\subsection{Docker}

O Docker\footnote{Disponível em https://www.docker.com/} é uma plataforma de automação que implanta as aplicações em espaços isolados chamados de \textit{contêineres}, possibilitando as executarem as aplicações de forma mais ágil. O objetvo é criar múltiplos ambientes dentro de um mesmo servidor, acessíveis externamente.

Segundo Matthias e Kane, no livro Primeiros Passos com Docker, essa ferramenta a arquitetura do software de modo a tornar o armazenamento ainda mais robusto. A essência no Docker é acomodar os serviços ou aplicações em contêineres atômicos ou descartáveis. Em todo o processo de desenvolvimento um contêiner pode ser excluído e ser restaurado se assim surgir a necessidade, isso torna a dinâmica de entrega e teste extremamente flexível.\cite{mattiaskane}

O Docker é uma plataforma Open Source, que não pode ser confundido com um ambiente tradicional de virtualização. No Docker fazemos uso de recurso isolados que utilizam bibliotecas do kernel em comum, isso porque é presente nessa ferramenta o Linux Containers (LXC).

O mesmo permite empacotar a aplicação ou um ambiente em um contêiner e movê-lo para qualquer outro host que possua o Docker instalado, tornando assim uma ferramenta portável. A grande vantagem disso é que não há necessidade de reconfigurar todo o ambiente novamente, visto que todo ele é movido, reduzindo acentuadamente o tempo de deploy de uma infraestrutura.


A proposta do Docker é exatamente essa. Não há a necessidade de subir várias máquinas virtuais, tão somente precisa-se de uma máquina, e será possível executar várias aplicações sem conflitos entre elas. Todas as dependências, bibliotecas e recursos necessários especificamente a cada software, serão disponibilizados no contêiner. Dessa fora, não é necessário instalar novamente todos os serviços em cada ambiente, diminui-se o uso de recursos e mantém as configurações da aplicação isoladas de outros softwares, evitando conflitos. \cite{scampini}

Essa plataforma não pode ser confundida com um ambiente virtualizado, visto que em cenários que utilizam máquinas virtuais há a presença de uma camada intermediária de sistema operacional entre o host e as aplicações, no Docker essa é desnecessária pois ele não utiliza kernel, tornando independente quanto a nível de disco, memória e processamento.\cite{mouat}

A infraestrutura no Docker também é replicável, é possível criar imagens predefinidas e disponibilizá-las em ambientes de desenvolvimento, teste, homologação e produção para aplicações.\cite{mattiaskane}

\subsubsection{Vantagens:}
Podemos elencar alguns ganhos consideráveis na utilização do Docker \cite{scampini} :
\begin{itemize}
\item Empacotamento de software otimizando o uso das habilidades dos desenvolvedores;
\item Empacotamento de aplicação de software com todos os arquivos e dependências necessárias para determinada aplicação.
\item Utilização de artefatos empacotados que possibilitem a passagem pelo teste e produção sem necessidade de recompilação.
\item Uso de softwares sem onerar recursos demasiados, visto que o contêiner é apenas um processo que se comunica diretamente com kernel do Linux.
\end{itemize}

\begin{figure} [htb]
	\centering

	\includegraphics[width=0.8\linewidth]{imagens/dockerXvm}
	\caption{Comparativo Docker x Máquinas Virtuais}
	Fonte: Bright Computing\footnotemark
	\label{fig:dockerXvm}

	\footnotetext{Disponível em: http://www.brightcomputing.com/blog/containerization-vs.-virtualization-heres-our-blog-smackdown}
\end{figure}
	


\subsection{Puppet}
É uma ferramenta de código livre para gestão de configurações. A ideia central do Puppet\footnote{Disponível em https://www.puppet.com/} é a administração de diversas máquinas físicas ou virtuais, onde a configuração é centralizada em um único nó e então distribuídas por diversos nós na rede, assim, gerenciando configurações, automatizando instalação de pacotes e facilitando o estabelecimento de normas e auditoria.\cite{walberg2008automate}

A ferramenta está disponível em duas versões: Puppet Enterprise\footnote{Disponível em: https://puppet.com/products/puppet-enterprise} (com suporte pago), e a Open Source Puppet\footnote{Disponível em: https://puppet.com/download-open-source-puppet} (código aberto). É uma ferramenta bem utilizada pela comunidade e por isso existem muitos módulos desenvolvidos, empresas como McAfee e Nasa fazem uso dela.

O Puppet utiliza SSH para a conexão aos hosts, esse é um ponto positivo se olharmos pela ótica que em alguns cenários não é possível a instalação de agentes ou em situações onde o agente consome uma fatia considerável de memória e cpu.

\begin{table}[h]
	\centering

	\begin{tabular}{|p{3.0cm}|p{3.0cm}|p{3.0cm}|p{3.0cm}|}
		\cline{1-4}
		
		  & Ansible &Puppet & Chef \\ % Note a separação de col. e a quebra de linhas
		\hline                               % para uma linha horizontal
		Linguagem do Script & YAML        & Custom DSL baseado em Ruby & Ruby\\ \cline{1-4}
		
		Infraestrutura & Máquina controladora aplica configuração em nós via SSH  & Puppet Master sincroniza configuração em nós Puppet & Chef Workstation empurra configuração para o servidor Chef onde os nós Chef serão atualizados  \\ \cline{1-4}
		
		Softwares especialiados requeridos para nós & Não & Sim & Sim\\ \cline{1-4}
		
		Fornece ponto de controle centralizado & Não, qualquer computador pode ser controlado &  Sim, via  Puppet Master & Yes, via servidor Chef \\ \cline{1-4}
		
		Terminologia de Scripts & Playbook/Roles & Manifests/ Módulos & Receitas/ Cookbooks \\ \cline{1-4}
		
		Ordem de Execução das Tarefas & Sequencial & Não sequencial & Sequencial \\ \cline{1-4}         
		
	\end{tabular}
	\caption{Comparativo entre Ferramentas de Gerenciamento de Configurações}
	Fonte: Digital Ocean\footnotemark
\end{table}
	\footnotetext{Disponível em: https://www.digitalocean.com/}


\subsection{Ansible}

O Ansible\footnote{Disponível em https://www.ansible.com} foi criado em 2012, por Michael DeHann, basicamente essa ferramenta gerencia configurações e orquestra tarefas. Nesse sentido ele implementa módulos para nós sobre SSH. Os módulos são distribuídos nos nós temporariamente que realizam a comunicação com a máquina de controle (assim com em ferramentas concorrentes) por meio de um protocolo JSON\footnote{Disponível em: http://jsonapi.org/}.

O diferencial do Ansible em relação as demais ferramentas que se propõe ao mesmo objetivo, é que ele atua com uma arquitetura cliente-servidor, sem agente,  isso que dizer que os nós não são necessários para a instalação dos daemons para comunicação com a máquina controle. Isso resulta em diminuição de carga na rede. 

Os objetivos mais notáveis do Ansible é tornar a experiência do usuário muito mais simples e fácil, e investir massivamente na segurança e confiabilidade utilizando o OpenSSH como condutor de dados. É uma linguagem construída temo como parâmetro a auditabilidade, ou seja, possibilitar ao analista o acompanhamento de todas as etapas, rotinas e dados circulados dentro de uma arquitetura definida.\cite{marcelocosta}

Nessa ferramenta ainda encontramos o uso de um arquivo de inventário, denominado "hosts", que definem quais nós serão gerenciados, que é simplesmente um arquivo de texto que lista os nós individualmente ou agrupados ou até um arquivo executável que constrói um inventário de hosts. É uma opção de alta confiabilidade e segurança, e isso se fundamenta pelo uso do Secure Shell. Possui ainda fácil usabilidade, no entanto, não deixa a desejar em qualquer aspecto se comparado a soluções concorrentes.

Na forma tradicional de trabalho o Ansible faz o upload do código que deve ser executado nas estações clientes, é então executado, retorna o resultado da execução e após isso é removido dos clientes.

Quando usamos a ótica DevOps esse tipo de fluxo é modificado, fazendo uso do protocolo NETCONF\footnote{Disponível em: https://tools.ietf.org/html/rfc6241} (RFC 6241), onde é possível o envio de comandos aos componentes e receber o retorno da aplicação.

\subsubsection{Componentes}
O Ansible é estruturado pela composição dos seguintes elementos:

\begin{itemize}
	\item Playbooks\footnote{Disponível em https://docs.ansible.com/ansible/playbooks.html}: arquivos de configuração, implementação e linguagem de orquestração do Ansible.
	\item Agentless\footnote{Disponível em https://dbruno.ansible.com/ansible/}: É descartado o uso de agente nos servidores a serem monitorados. Isso deve-se à utilização de OpenSSH para definir o estado atual do ambiente, adaptando-se se acaso estiver em desconformidade com a configuração no playbook.
	\item Módulos\footnote{Disponível em https://docs.ansible.com/ansible/modules.html}: São as tarefas executadas de fato. Os módulos são ditos como "plugins de tarefas" e por isso são eles que realizam as atividades pertinentes.
	\item Inventário\footnote{Disponível em https://dbruno.ansible.com/ansible/}: Armazena e controla informações sobre os grupos de hosts.
\end{itemize}

\subsubsection{Vantagens}
Podemos apontar alguns ganhos consideráveis na utilização do Ansible[10]:
\begin{itemize}
	\item Não há a necessidade na instalação de agentes nos servidores a serem gerenciados;
	\item Gerencia paralela e simultaneamente de forma orquestrada; 
	\item Simples configuração e bem estruturado;
	\item Desenvolvimento em diversas linguagens.
\end{itemize}

\subsection{GitHub }
É um dos serviços web mais difundidos. Com essa ferramenta é possível hospedar projetos e aplicações, trabalhando com controle de versionamento.

O Github\footnote{Disponível em https://www.github.com} funciona por meio de repositórios, que funcionam como pastas e dentro destas outras pastas que comportam os arquivos de diversas extensões e linguagens. Ainda oferece a possibilidade de contribuir com um determinado para o desenvolvimento de determinado projeto. Além, e permitir o acesso múltiplo por diversos desenvolvedores de um mesma empresa, permitindo que vários programadores trabalhem simultaneamente em uma mesma aplicação ou arquivo.

\begin{figure}
	\centering
	\includegraphics[width=0.7\linewidth]{git}
	\caption{Fluxo commit e request}
	Fonte: Cloud Turbine\footnotemark
	\label{fig:git}
\end{figure}
	\footnotetext{Disponível em: https://www.cloudturbine.com/using-github-and-git/}

\subsection{Jenkins}

Essa ferramenta multiplataforma e funciona como um servidor destinado a integração contínua que automatiza a execução de tarefas, possui código aberto e permite ao usuário total liberdade em sua operação.

Basicamente o Jenkins\footnote{Disponível em https://www.jenkins.io} atua em um conceito de integração contínua, onde o objetivo é agilizar tarefas que tradicionalmente demandam ]um tempo de excecução mais prolongado como compilação do projeto e execução de testes automatizados. Cada interação é analisada por um build automatizado a fim de detectar possíveis erros de integração, isso permite que testes sejam realizado com o objetivo de reduzir problemas de updates e tornando o software mais coeso.\cite{atalay}

O Jenkins pode ser integrado ao Git, SVN, CVS, Maven, entre outros. É importante ressaltar que um ponto forte do Jenkins é a difusão entre a comunidade. Ele ainda possui mais de 1000 plugins disponíveis e utilizado por várias empresas de desenvolvimento de softwares.

Um recurso interessante é o uso de Forks (recurso do GitHub), que permitem a criação de cópias de um determinado projeto e trabalha nesse sem a preocupação de afetar em algum ponto a aplicação original. É possivel ainda, adicionar testes de desempenho e balanceamento na integração contínua, permitindo a análise de risco e a reduzir eventuais quedas de performance no momento em que um novo recurso é adicionado ou a correção de um erro presente.

\subsubsection{Vantagens}
\begin{itemize}
\item Builds periódicos;
\item Testes automatizados;
\item Possibilita análise de código;
\item Identificar erros mais cedo;
\item Fácil de operar e configurar;
\item Comunidade ativa;
\item Integra outras ferramentas por meio de plugins nativos.
\end{itemize}



% ---
% Capitulo de METODOLOGIA
% ---




\chapter{Metodologia}\label{cap:metodologia}

No âmbito desse estudo, é proposta uma arquitetura para implantação de hospedagem de aplicações externas. Assim, o levantamento de informações consistiu em levantar um conjunto de requisitos essenciais para apoiar o modelo apresentado.

Para a eleição das ferramentas escolhidas foi levando em conta, a sua disponibilização de código open source, a flexibilidade no uso e confiabilidade nos resultados, as suas vantagens e desvantagens, a aceitação junto a comunidade científica e a documentação disponível, seja por meio de livros ou artigos publicados ou por contribuições junto a comunidade web.

O processo de eleição das ferramentas teve como ponto inicial estar em consonância com atuabilidade do mercado de TI e ter continua contribuição da comunidade web. Sendo assim, o primeiro critério a ser levado em conta foi como a ferramenta daria suporte às diversas linguagens de programações disponíveis. Outro fator relevante está atrelado a maneira de como o produto pode ser adquirido, mais diretamente ao custo necessário para fazer uso da ferramenta, a premissa é que as mesmas teriam versão open source (gratuita) que  dessem suporte às necessidades da arquitetura proposta.

Dessa forma, alguns requisitos foram considerados na escolha, os pontos citados a seguir nortearam o estudo: 

\begin{itemize}
	
	\item Extensibilidade: tomando como fato as rápidas mudanças no cenário tecnológico, a ferramenta deveria poder dar suporte a diferentes linguagens e integrações;
	\item Usabilidade: a ferramenta deveria ser de fácil compreensão e fornecer uma boa experiência ao usuário;
	\item Segurança: é essencial que critérios de segurança sejam inerentes à ferramenta, definição de papéis e usuários.
	
\end{itemize}


\section{Ferramentas Utilizadas}

Tomando por base os critérios mencionados no item 3, passou-se a definição das ferramentas, são elas:

\begin{itemize}
	
\item VirutalBox: Criação de máquinas virtuais para testes. Serão construídas máquinas virtuais para execução dos hosts clientes e servidores;
\item GitHub: Versionamento e repositório. Esse serviço receberá os códigos das aplicações dos desenvolvedores;
\item Ansible: Gerenciamento de configuração e orquestração da rotina. Essa ferramenta será responsável por administrar as rotinas dentro de todo o processo de deploy. Será executado em uma máquina virtual com Sistema Operacional Ubuntu Linux 64 bits.
\item Jenkins: Automatização da execução de tarefas e testes de implantação. O Jenkins será executado em uma máquina virtual com Sistema Operacional Ubuntu Linux 64 bits.
\item Docker: Contêineres para armazenamento das aplicações e testes. O Docker será executado em uma máquina virtual com Sistema Operacional Ubuntu Linux 64 bits.

\end{itemize}





\chapter{Resultados}\label{cap:resultados}

Após análise das ferramentas estudadas e posteriormente eleitas as que melhores se adequaram ao presente estudo, partiu-se para a elaboração e concepção do modelo proposto para entrega de aplicação externa com base em uma visão DevOps.

A figura 5 representa a proposta de modelo de arquitetura para aplicação externa pretendida nesse estudo. O processo que dará início à sequencia de testes e deploy, começará com a requisição do usuário (desenvolvedor) para adquirir um repositório com intuito de armazenar uma determinada aplicação web. Isso quer dizer, que o programador terá acesso a um formulário de requisição (via sistema web) para armazenamento do seu código e após analisado, será retornado ao mesmo o link desse repositório a fim de que o mesmo possa hospedar a sua aplicação. Os códigos e alterações serão enviados ao GitHub. Depois de enviar o código, o Jenkins receberá continuamente os códigos e executará os testes. Sendo os testes realizados com sucesso e tendo retorno positivo, sem erros, o Jenkins realizará o deploy no servidor através do Docker.

\begin{figure} [htb]
	\centering
	\includegraphics[width=1.05\linewidth]{imagens/MODELODEPLOY}
\caption{Representação da Arquitetura Proposta}
Fonte: Acervo próprio
	\label{fig:modelodeploy}
\end{figure}


Considerando o modelo apresentado na figura 5, temos como sequenciamento mais detalhado, as etapas a seguir, absorvendo desde o início de uma requisição até a disponibilização da aplicação em servidor de produção.

\begin{enumerate}
	\item No ponto de partida, é pre-requisito a existência de conteúdo codificado em linguagem de programação, que permita o versionamento do projeto.
	\item O desenvolvedor faz a requisição de um novo repositório através de um formulário, que contenha as informações pertinentes, tais como, chave ssh, membros do projeto e nome do mesmo;
	\item A requisição é recebida pelo provedor que verifica os dados de solicitação, avalia a disponibilidade de repositório;
	\item O provedor valida a disponibilidade e cria um novo repositório no GitHub;
	\item O link do repositório é retornado ao usuário juntamente com sua chave de autenticação;
	\item O desenvolvedor envia os commits e o GitHub gera uma notificação de alteração ao Jenkins;
	\item Jenkins recebe a notificação (Git Fetch – Git Merge – Templates Ansible – Dry Run – Commit Check);
	\item O Jenkins realiza teste jundo ao contêiner Docker destinado para tanto;
	\item Jenkins gera um histórico (Notificação aos Administradores);
	\item Administrador analisa os logs de testes;
	\item GitHub gera nova notificação ao Jenkins;
	\item Jenkins gera novo release e gera nova tag de versão;
	\item Estando as alterações aprovadas o Jenkins realiza deploy da aplicação em um contêiner Docker destinado a produção;
\end{enumerate}

















\chapter{Conclusão}\label{cap:conclusao}

Este trabalho concentrou-se na apresentação dos conceitos da visão DevOps e analogia com o modelo tradicional de entrega de um software, onde ao decorrer do estudo analisou-se diversas ferramentas disponíveis no mercado e elegeram-se as que mais se adequavam a necessidade da implementação de um modelo de entrega para aplicação externa com o conceito de deploy automatizado.

O deploy automatizado fornece ao desenvolvedor e profissional de TI um ambiente simplificado para entrega de uma aplicação externa, tornando o gestor de tecnologia um provedor de serviço.

O presente modelo tem como foco contribuir para o aprimoramento no processo de entrega de software. Nesse sentido o cenário atual ainda lida com o processo de deploy baseado no modelo tradicional "cascata", com o estudo realizado nesse projeto é possível otimizar o planejamento do processo da entrega da aplicação para produção utilizando recursos automatizados e gratuitos (open source) e uma visão DevOps. É possível economizar não só tempo como recursos humanos e financeiros que são fundamentais para a conclusão do processo.

Para continuidade deste projeto, objetiva-se a implementação de um cenário funcional utilizando máquinas virtuais, como expõe-se a seguir:

\begin{enumerate}
	\item Implementar o cenário apresentado neste projeto;
	\item Validar o modelo considerando diversas situações;
	\item Implementar o modelo visando ambientes críticos e cenários de risco.
	\item Implantar a solução afim de gerar um estudo de caso, visando verificar a aceitação da mesma.
\end{enumerate}


% ----------------------------------------------------------
% ELEMENTOS PÓS-TEXTUAIS
% ----------------------------------------------------------
\postextual
% ----------------------------------------------------------

% ----------------------------------------------------------
% Referências bibliográficas
% ----------------------------------------------------------
\bibliography{bibliografia}
.\\
SATO, Danilo. DEVOPS Na Prática: Entrega de Software Confiável e Automatizada. São Paulo: Casa do Código, 2016.
\\ \\ 
MATTHIAS, Karl e KANE, Sean. Primeiros Passos com Docker: Usando Contêineres em Produção. São Paulo: Novatec, 2015.
\\ \\
IBM. O que é DEVOPS? 2013. Disponível em: $<$https://www.ibm.com/developerworks/\\ community/blogs/rationalbrasil/entry/o\_que\_devops?lang=en$>$. Acesso em: 12 fev 2018.
\\ \\
GARTNER, Group. Glossário de Ti: DEVOPS. Disponível em:
$<$https://www.gartner.com/ \\it-glossary/devops$>$. Acesso em: 13 fev 2018.
\\ \\
GARTNER, Group. Newsroom: Gartner Says By 2016, DevOps Will Evolve From a Niche to a Mainstream Strategy Employed by 25 Percent of Global 2000 Organizations. Disponível em: $<$https://www.gartner.com/newsroom/id/2999017$>$. Acesso em 13 fev 2018.
\\ \\
GAEA. 4 Motivos para Usar a Cultura DevOps.Disponível em: $<$https://gaea.com.br/4-motivos-para-usar-a-cultura-devops/\#$>$. Acesso em: 13 fev 2018.
\\ \\ 
COSTA, Bruna Lima. DevOps, quem usa? como adotar? o que usar?. Disponível em: $<$https://www.primeinf.com.br/devops-quem-usa-como-adotar-o-que-usar/$>$. Acesso em 13 fev 2018
\\ \\ 
HASHIMOTO, Mitchell. Atlas. Disponível em: $<$https://www.hashicorp.com/blog/atlas-announcement$>$. Acesso em 14 fev 2018.
\\ \\
DIEDRICH, Cristiano. O que é Docker? Disponível em: $<$https://www.mundodocker.com.br/o-que-e-docker/$>$. Acesso em 14 fev 2018.
\\ \\
COSTA, Marcelo. Ansible: uma nova opção para gerenciamento de configuração. Disponível em: $<$https://www.infoq.com/br/news/2013/04/ansible1.1$>$. Acesso em 14 fev 2018.
\\ \\

% ----------------------------------------------------------
% Glossário
% ----------------------------------------------------------
%
% Consulte o manual da classe abntex2 para orientações sobre o glossário.
%
%\glossary

%---------------------------------------------------------------------
% INDICE REMISSIVO
%---------------------------------------------------------------------
\phantompart
\printindex
%---------------------------------------------------------------------

\end{document}
\grid
