% ----------------------------------------------------------
% ELEMENTOS PRÉ-TEXTUAIS
% ----------------------------------------------------------
% \pretextual

% ---
% Capa
% ---
\imprimircapa
% ---

% ---
% Folha de rosto
% (o * indica que haverá a ficha bibliográfica)
% ---
\imprimirfolhaderosto
% ---

% ---
% Inserir folha de aprovação
% ---

% Isto é um exemplo de Folha de aprovação, elemento obrigatório da NBR
% 14724/2011 (seção 4.2.1.3). Você pode utilizar este modelo até a aprovação
% do trabalho. Após isso, substitua todo o conteúdo deste arquivo por uma
% imagem da página assinada pela banca com o comando abaixo:
%
% \includepdf{folhadeaprovacao_final.pdf}
%
\begin{folhadeaprovacao}

  	
  	
  	\begin{center}
  		\includegraphics[width=1\textwidth]{imagens/unitins.png}
  		\ABNTEXchapterfont\Large   CURSO DE SISTEMAS DE INFORMA{\c{C}}{\~{A}}O
  		
  		\par
  		\vspace*{1cm}     
  		{\ABNTEXchapterfont\bfseries\large \expandafter\MakeUppercase{\imprimirtitulo}  \vspace*{1cm}    }
  		\par
  		{\large \expandafter\MakeUppercase{\imprimirautor}}
  		%\vspace*{\fill}
  		\par
  		\vspace*{1cm}     
  		\hspace{.45\textwidth}
  		\begin{minipage}{.5\textwidth}
  			\small\imprimirpreambulo
  			
  		\end{minipage}%
  	%	\vspace*{\fill}
  	\end{center}
  
        
  
   \assinatura{\textbf{\imprimirorientador} \\ Orientador} 
   \assinatura{\textbf{Professor} \\ Convidado 1}
   \assinatura{\textbf{Professor} \\ Convidado 2}
   %\assinatura{\textbf{Professor} \\ Convidado 3}
   %\assinatura{\textbf{Professor} \\ Convidado 4}
      
   \begin{center}
    \vspace*{0.5cm}
    {\large\imprimirlocal}
    \par
    {\large\imprimirdata}
    \vspace*{1cm}
  \end{center}
  
\end{folhadeaprovacao}
% ---

% ---
% Dedicatória
% ---
\begin{dedicatoria}
   \vspace*{\fill}
   \centering
   \noindent
   \textit{ Este trabalho é dedicado à minha família, pelo apoio incondicional.} \vspace*{\fill}
\end{dedicatoria}
% ---

% ---
% Agradecimentos
% --- Es
\begin{agradecimentos}
À Deus, pela graça de permitir o meu crescimento no campo de estudo em que me sinto feliz e realizado.
Aos meus pais que sempre apostaram em minha capacidade.
À minha esposa que tem sido compreensiva nos momentos de dificuldades, e companheira nos desafios que surgem.
Aos meus amigos e colegas de universidade, que sempre contribuíram para o desenvolvimento do meu conhecimento e pelas parcerias nos projetos e estudos.


\end{agradecimentos}
% ---

% ---
% Epígrafe
% ---
\begin{epigrafe}
    \vspace*{\fill}
	\begin{flushright}
		\textit{``Não só isso, mas também nos gloriamos \\
			nas tribulações, porque sabemos que a tribulação\\
			produz perseverança; a perseverança, um caráter \\
			aprovado; e o caráter aprovado, esperança. \\
			E a esperança não nos decepciona, porque Deus \\
			derramou seu amor em nossos corações, por meio \\
			do Espírito Santo que Ele nos concedeu.``\\
			(Bíblia Sagrada, Romanos 5:3-5)}
			
	\end{flushright}
\end{epigrafe}
% ---

% ---
% RESUMOS
% ---

% resumo em português
\setlength{\absparsep}{18pt} % ajusta o espaçamento dos parágrafos do resumo
\begin{resumo}
Este projeto tem como objetivo apresentar um estudo sobre as diversas ferramentas de automação da entrega de software, visando estabelecer uma proposta de cenário que viabilize esse processo dentro de uma cultura DevOps. O cenário deve funcionar como um provedor de serviços sobre plataforma ágil.
 

 \textbf{Palavras-chaves}: Automação, DevOps, Modelo Ágil.
 
\end{resumo}

% resumo em inglês
\begin{resumo}[Abstract]
 \begin{otherlanguage*}{english}
This project aims to present a study about the various automation tools of software delivery, aiming to establish a scenario proposal that will make this process feasible within a DevOps culture. The scenario should act as a service provider on agile platform.


   \vspace{\onelineskip}
 
   \noindent 
   \textbf{Key-words}: Automation, DevOps, Agile Model.
 \end{otherlanguage*}
\end{resumo}


% ---
% inserir lista de ilustrações
% ---
\pdfbookmark[0]{\listfigurename}{lof}
\listoffigures*
% ---

% ---
% inserir lista de tabelas
% ---
% \newpage
% \pdfbookmark[0]{\listtablename}{lot}
% \newpage
% \listoftables*
% \cleardoublepage
% ---

% ---
% inserir lista de abreviaturas e siglas
% ---
\begin{siglas}
  \item DevOps - Development and Operational (Desenvolvimento e Operacional).
  \item AWS - Amazon Web Services
  \item SSH - Secure Shell
  \item NETCONF - Network Configuration
\end{siglas}
% ---


% ---
% inserir o sumario
% ---
\pdfbookmark[0]{\contentsname}{toc}
\tableofcontents*
\cleardoublepage
% ---

